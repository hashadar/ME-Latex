\documentclass[class=report, crop=false, 12pt,a4paper]{standalone}
\usepackage{enumitem}
\usepackage{float}
\usepackage[normalem]{ulem}
\usepackage{graphicx}
\usepackage{amsmath}
\usepackage{amssymb}
\usepackage{siunitx}
\usepackage{commath}
\usepackage{tikz}
\usetikzlibrary{positioning, fit, calc}   
\tikzset{block/.style={draw, thick, text width=3cm ,minimum height=1.3cm, align=center},   
line/.style={-latex}     
}  
\begin{document}
\section*{Exercise 2 - Fluid tutorial group B}
\subsection*{i}
\begin{gather}
  \psi = A r^n \sin(n \theta)\\
  u_\theta = -\frac{\partial \psi }{\partial r}  = \frac{1}{r} \frac{\partial \phi }{\partial \theta } \\
  u_\theta = -nAr^{n-1} \sin(n\theta)\\
  -nAr^{n-1} \sin(n\theta) = \frac{1}{r} \frac{\partial \phi }{\partial \theta }\\
  \int_{}^{} \left( -nAr^{n} \sin(n\theta) \right) \,\mathrm{d}\theta = \int_{}^{} \left( \frac{\partial \phi }{\partial \theta } \right) \,\mathrm{d}\theta \\
  \phi = Ar^n\cos(n\theta) + c 
\end{gather}
\subsection*{ii}
\begin{gather}
  \nabla \cdot \vec{V} = \frac{\partial^2 \phi}{\partial r^2} + \frac{1}{r} \frac{\partial \phi}{\partial r} + \frac{1}{r^2}\frac{\partial^2\phi}{\partial \theta^2} = 0 \\
  \frac{\partial \phi}{\partial r} = nAr^{n-1}\cos(n\theta) \\
  \frac{\partial^2 \phi}{\partial r^2} = n(n-1)Ar^{n-2}\cos(n\theta)\\
  \frac{\partial\phi}{\partial \theta} = -nAr^n\sin(n\theta)\\
  \frac{\partial^2\phi}{\partial \theta^2} = -n^2Ar^n\cos(n\theta)\\
  n(n-1)Ar^{n-2}\cos(n\theta) + nAr^{n-2}\cos(n\theta) -n^2Ar^{n-2}\cos(n\theta) = 0\\
  Ar^{n-2}\cos(n\theta) \left[ n(n-1) + n - n^2 \right] = 0\\
  n^2 - n + n - n^2 = 0 \\
  \therefore Ar^{n-2}\cos(n\theta) \times 0 = 0 \\
  0 = 0 
\end{gather}
Continuity equation satisfied.
\subsection*{iii}
\begin{align}
  180 &= 120n\\
  n &= \frac{3}{2}
\end{align}
\subsection*{iv}
\begin{gather}
  p + \frac{1}{2} \rho \left( u_r^2 + u_\theta^2 \right) = c\\
  u_r = \frac{\partial \phi}{\partial r} = nAr^{n-1}\cos(n\theta)\\
  u_{\theta} = \frac{1}{r} \frac{\partial \phi}{\partial \theta} = - n A r^{n-1} \sin(n \theta)\\
  \textrm{A: } 20000 + \frac{1}{2}(1000)\left(\frac{9}{4} + 0\right) = 21125\\
  \textrm{B: }p + \frac{1}{2}(1000)\left(\frac{9}{4} + \frac{9}{4}\right) = 21125\\
  p = 18875 \ \si{\pascal}
\end{gather}
\subsection*{v}
\begin{gather}
  \psi = Ar^n \sin(n\theta)\\
  \psi = 1\times 2^{\frac{3}{2}} \sin(\frac{3}{2}\times 30) =2\\
\end{gather}
\end{document}