\documentclass[class=report, crop=false, 12pt,a4paper]{standalone}
\usepackage{enumitem}
\usepackage{multicol}
\usepackage{etoolbox}
\AtBeginEnvironment{quote}{\singlespacing\small}
\usepackage{setspace}
\onehalfspacing
\usepackage{graphicx}
\usepackage{float}
\usepackage{amsmath}
\usepackage{amssymb}
\usepackage{mathtools}
\usepackage{siunitx}
\sisetup{detect-all}
\begin{document}
\section{Fluids formulas}
\subsection{Density}
\begin{equation*}
  v = \frac{1}{\rho}
\end{equation*}
\subsection{Specific weight}
\begin{equation*}
  \gamma = \rho g
\end{equation*}
\subsection{Specific gravity}
\begin{equation*}
  SG = \frac{\rho}{\rho_{H_2O @ 4\si{\degree}C}}
\end{equation*}
\subsection{Ideal gas law}
\begin{align*}
  P &= \rho R T\\
  Pv &= RT\\
  PV &= n R_u T
\end{align*}
\subsection{Kinematic viscosity}
\begin{equation*}
  v = \frac{\mu}{\rho}
\end{equation*}
\subsection{Surface tension}
\begin{equation*}
  \sigma = \frac{F}{L}
\end{equation*}
\subsection{Capillary action}
\begin{equation*}
  h = \frac{2 \sigma \cos \theta}{\gamma R}
\end{equation*}
\subsection{Centre of mass}
\begin{align*}
  \bar{x} &= \frac{\int_m (x) dm}{m}\\
  &= \frac{\int_m (x)dm}{\int_m dm}
\end{align*}
\subsection{Stable equilibrium}
The centre of gravity is directly below the centre of buoyancy.
\subsection{Reynolds Number}
\begin{equation*}
  Re = \frac{\rho L u}{\mu}
\end{equation*}
\[ Re < 2000 \rightarrow \textrm{ Laminar} \]
\[ Re > 2000 \rightarrow \textrm{ Turbulent} \]
\subsection{Material derivative}
\begin{align*}
  \frac{D}{Dt}() &= \frac{\partial}{\partial t}() + u \frac{\partial()}{\partial x} + v \frac{\partial ()}{\partial y} + w\frac{\partial ()}{\partial z}\\
  \underline{a} = \frac{D}{Dt}(\underline{v}) &= \frac{\partial v}{\partial t} + u\frac{\partial v}{\partial x} + v\frac{\partial v}{\partial y} + w \frac{\partial v}{\partial z} 
\end{align*}
\subsection{Streamline coordinates}
\begin{align*}
  \underline{a} = \frac{Dv}{Dt} &= a_s\hat{s} + a_n\hat{n}\\
  &= v\frac{\partial v}{\partial s} \hat{s} + \frac{v^2}{R} \hat{n}\\
  a_s &= \frac{-v\sin \theta - \frac{\partial P}{\partial s}}{\rho} = v\frac{\partial v}{\partial s}\\
  a_n &= \frac{-v\cos \theta -\frac{\partial P}{\partial n}}{\rho} = \frac{v^2}{R}
\end{align*}
\subsection{RTT}
\begin{align*}
  \left( \frac{DB}{Dt} \right)_{sys} &= \frac{\partial}{\partial t}\int_{cv}(\rho b) d\forall + \int_{cs} (\rho b(\underline{v} \cdot \underline{\hat{n}})) dA\\
  \textrm{When B=M} \rightarrow 0 &= \frac{\partial}{\partial t}\int_{cv}(\rho) d\forall + \int_{cs} (\rho(\underline{v} \cdot \underline{\hat{n}})) dA\\
  \textrm{When \underline{B} = m\underline{v}} \rightarrow \sum F_{sys} &= \frac{\partial}{\partial t}\int_{cv}(\rho \underline{v}) d\forall + \int_{cs} (\rho \underline{v} (\underline{v} \cdot \underline{\hat{n}})) dA\\
\end{align*}
\subsection{Propeller stages}
\begin{enumerate}
  \item Bernoulli equation between 1 and 2. \[ P_1 = P_2 +0.5\rho (v_2^2 - v_1^2) \]
  \item Bernoulli equation between 3 and 4. \[ P_4 = P_3 +0.5\rho (v_3^2 - v_4^2) \]
  \item \( P_1 = P_4 \). \[ \therefore \Delta P = P_3 - P_2 = 0.5 \rho (v_4^2 - v_1^2) \]
  \item \( F=\Delta P \times A \). \[ \therefore F_{thrust} = \Delta P \times A = 0.5 \rho A (v_4^2 - v_1^2) \]
  \item Apply momentum equation.
    \begin{align*}
      \sum F_x &= \int_{inlet}\rho v_1 (-v_1)dA + \int_{outlet}\rho v_4 (v_4)dA\\
      &= -\dot{m}_1 v_1 + \dot{m}_2 v_4\\
      &= \dot{m}(v_4 - v_1) \textrm{ (assume steady state, no viscous force, neglect g)}
    \end{align*}
  \item At the propeller, \( \dot{m} = \rho A_P v_p = \rho A v_p \). \[ \therefore F_{\textrm{thrust}} = \rho A v_P(v_4 - v_1) \]
  \item \( 0.5 \rho A (v_4^2 - v_1^2) = \rho A v_p (v_4 - v_1) \). \[ \therefore v_P = 0.5 (v_4 + v_1) \]
  \item Work out \(v_P\) and use it to work out thrust.
  \item \( P = F \times v \).
\end{enumerate}
\subsection{Moving control volume}
\[ \underline{v} = \underline{v}_{CV} + \underline{W} \]
Where \( \underline{v} \) is the absolute velocity, \( \underline{v}_{CV} \) is the velocity of the control volume and \( \underline{W} \) is the velocity relative to the control volume.
\subsubsection{Conservation of mass}
\[ 0 = \frac{\partial}{\partial t}\int_{CV} \left( \rho \right) d\forall + \int_{CV} \left( \rho (\underline{W} \cdot \underline{\hat{n}}) \right) dA \]
\subsubsection{Conservation of momentum}
\[ \sum \underline{F}_{sys} = \int_{CV} \left( \rho \underline{W} (\underline{W} \cdot \underline{\hat{n}}) \right) dA \]
When steady state and in an inertial reference frame.
\subsection{Energy equation}
\[ \left[ \sum \dot{Q}_{\textrm{net in}} + \sum \dot{W}_{\textrm{net in}} \right]_{CV} = \frac{\partial}{\partial t} \int_{CV} \left( e \rho \right) d \forall + \int \left( e \rho (\underline{v} \cdot \underline{\hat{n}}) \right) dA \]
Or,
\[ \dot{Q}_{\textrm{net in}} + \dot{W}_{\textrm{shaft net in}} =\frac{\partial}{\partial t} \int_{CV} \left( e \rho \right) d \forall + \int_{CS} \left( u + \frac{P}{\rho} + \frac{v^2}{2} + gz \right) \rho (\underline{v} \cdot \underline{\hat{n}}) dA \]
When steady flow and only one stream,
\begin{multline*}
  \dot{Q}_{\textrm{net in}} + \dot{W}_{\textrm{shaft net in}} =\\ \dot{m} \left[ u_{out} - u_{in} + \left( \frac{P}{\rho} \right)_{out} - \left( \frac{P}{\rho} \right)_{in} + \frac{v_{out}^2 - v_{in}^2}{2} + g(z_{out}- z_{in})  \right]
\end{multline*}
In the Bernoulli form:
\begin{align*}
  \frac{P_{out}}{\rho} + \frac{v_{out}^2}{2} + gz_{out} &= \frac{P_{in}}{\rho} + \frac{v_{in}^2}{2} + gz_{in} + W_{\textrm{shaft net in}} - \textrm{loss, which is:}\\
  \textrm{loss} &= u_{out} - u{in} - q_{\textrm{net in}}
\end{align*}
\subsection{Shaft work efficiency}
\begin{align*}
  \eta &= \frac{W_{\textrm{shaft net in}} - \textrm{loss}}{W_{shaft net in}} 
  \textrm{Where loss} &= u_{out} - u_{in} - q_{\textrm{net in}}
\end{align*}
\subsection{Laminar flow and boundary layers}
\begin{align*}
  u_y(0) &= 0\\
  u_y(D) &= v\\
  \tau &= \mu \frac{du_y}{dx} \\ 
  \frac{\tau}{\mu}\int du &= \int du_y\\
  \frac{\tau}{\mu}x + k &= u_y\\
  \textrm{When } x=0, \ u_y &= 0 \rightarrow k = 0\\
  \therefore u_y &= \frac{\tau}{\mu} x\\
  u_y &= \frac{v}{D} x
\end{align*}
\subsection{Flow through porous media}
\[ \overline{u} = -C \frac{\partial P}{\partial x} \]
\subsection{Newtonian fluids}
\[ \overline{u} \propto \frac{1}{\mu} \] 
\subsection{Hagen-Poiseulle law}
\begin{align*}
  \Delta P &= \frac{dP}{dx} L\\
  \tau &= \mu \frac{du_y}{dx} = \mu \frac{du_y}{dr}\\
  F &= \tau \cdot 2\pi r \cdot L\\
  F &= \left( \frac{dP}{dx} \cdot L \right) \pi r^2\\
  u &= - \frac{1}{4\mu} \frac{dP}{dx}(R^2 - r^2)\\
  u_{mean} = \overline{u} &= \frac{Q}{\textrm{Area}} = -\frac{R^2}{8\mu}\frac{dP}{dx}\\
  u_{mean} &= -\frac{1}{4\mu} \frac{dP}{dx} R^2\\
  Q &= -\frac{\pi R^4}{8\mu}\frac{dP}{dx}\\
  \textrm{note } u_{mean} &= \overline{u} = \frac{u_{max}}{2}
\end{align*}
\subsection{Boundary layer}
\begin{itemize}
  \item Laminar region, thickness increases as \(x^{0.5}\).
  \item Turbulent region, thickness increases as \( x^{0.8} \)
  \item Boundary is taken to be at the contour where the velocity is 99\% of main flow.
\end{itemize}
\subsection{Displacement thickness}
\[ S^* = \int_0^\infty \left( 1 - \frac{u}{u_m} \right) dy \]
\subsection{Momentum thickness}
\[ \theta = \int_0^\infty \frac{u}{u_m} \left( 1 - \frac{u}{u_m} \right) dy \]
\end{document}