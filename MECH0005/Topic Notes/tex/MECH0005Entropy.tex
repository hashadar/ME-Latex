\documentclass[class=report, crop=false, 12pt,a4paper]{standalone}
\usepackage{enumitem}
\usepackage{multicol}
\usepackage{etoolbox}
\AtBeginEnvironment{quote}{\singlespacing\small}
\usepackage{setspace}
\onehalfspacing
\usepackage{graphicx}
\usepackage{float}
\usepackage{amsmath}
\usepackage{amssymb}
\usepackage{siunitx}
\sisetup{detect-all}
\begin{document}
\section{Entropy}
An important inequality that has major consequences in thermodynamics is the \emph{Clausius inequality}.
\[ \oint \frac{SQ}{T} \geq 0 \]
The cyclic integral of \( \frac{SQ}{T} \) is always less than or equal to zero. This inequality is valid for all cycles reversible or irreversible. \( \frac{SQ}{T} \) is the sum of all differential amounts of heat transfer to or from a system, divided by the temperature of the boundary.
\subsection{Proof of the Clausius inequality}
INSERT PROOF
\subsection{The increase of entropy principle}
Consider a cycle. It has two processes:
\begin{itemize}[noitemsep]
  \item Process 1-2: could be reversible or irreversible.
  \item Process 2-1L Internally reversible.
\end{itemize}
The Clausius inequality states:
\[ \oint \frac{SQ}{T} \geq 0 \]
Or,
\[ \int_1^2 \frac{SQ}{T} + \int_1^2\left( \frac{SQ}{T} \right)_{\textrm{int rev}} \geq 0 \]
\[ \int_1^2 \frac{SQ}{T} + (S_1 - S_2) \geq 0 \]
\[ S_2 - S_1 \leq \int_1^2 \frac{SQ}{T} \]
When written in the differential form:
\[ ds \leq \frac{SQ}{T} \]
Where T is the thermodynamic temperature at the boundary. SQ is the heat transferred between the system and surroundings. ds is the differential change in energy. When reversible \( ds = \frac{SQ}{T} \). When irreversible \( ds \leq \frac{SQ}{T} \). This equation shows that:
\begin{quote}
  \begin{center}
    Change in entropy of a closed system during an irreversible process is \emph{always greater} than the integral of \(\frac{SQ}{T}\) evaluated for that process.
  \end{center}
\end{quote}
\end{document}