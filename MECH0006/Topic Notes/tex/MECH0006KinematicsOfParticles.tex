\documentclass[class=report, crop=false, 12pt,a4paper]{standalone}
\usepackage{enumitem}
\usepackage{multicol}
\usepackage{etoolbox}
\AtBeginEnvironment{quote}{\singlespacing\small}
\usepackage{setspace}
\onehalfspacing
\usepackage{graphicx}
\usepackage{float}
\usepackage{siunitx}
\sisetup{detect-all}
\begin{document}
What is kinematics?
\begin{quote}
  \begin{center}
    Kinematics is used to describe the geometry of motion of an object in space, independent of the causes of that motion.
  \end{center}
\end{quote}

What is a particle?
\begin{quote}
  \begin{center}
    A particle is an object for which we do not need to worry about its rotation or the tendency of the forces that acts upon it which can cause rotation. Thus, we are not worries about our moment balance equation. A particle is a system idealised by being totally characterised by its position (as a function of time) and its fixed mass.
  \end{center}
\end{quote}

\section{Position, Velocity and Acceleration}
Consider a point mass \(M\) in 3D space with position \(P\) uniquely characterised by the vector \underline{r} from a fixed reference frame with origin 0. Next consider two adjacent points P and P' at the corresponding times of \(t\) and \(t + \Delta t\). The velocity of the point mass is defined as the limit of the change in position with respect to time given by:
\begin{equation}
  \underline{v} = \lim_{\Delta t \rightarrow 0}
\end{document}