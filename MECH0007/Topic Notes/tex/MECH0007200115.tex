\documentclass[class=report, crop=false, 12pt,a4paper]{standalone}

\begin{document}
What is dimension?
\begin{itemize}
  \item Any physical quantity
  \item Magnitude measured in units
\end{itemize}

What are primary dimensions and secondary dimensions?
\begin{itemize}
  \item Primary dimensions (fundamental) are ones that are independent and thus cannot be broken into constituent measurements
  \item Secondary dimensions (derived) are ones that can be expressed in terms of primary dimensions
\end{itemize}

What is an open system?
\begin{itemize}
  \item Allows mass and energy to cross its boundary
  \item Does not always mean there is a controlled volume
\end{itemize}

What is a closed system?
\begin{itemize}
  \item Only allows energy to cross its boundary
\end{itemize}

What is meant by the "surroundings" of a system?
\begin{itemize}
  \item The mass or region outside the system
\end{itemize}

What is a control volume?
\begin{itemize}
  \item May involve fixed, moving, real and imaginary boundaries (which must be drawn using dotted lines)
\end{itemize}

What is an intensive property?
\begin{itemize}
  \item Those that are independent of the mass of a system, such as temperature, pressure, and density.
\end{itemize}

What is an extensive property?
\begin{itemize}
  \item Those whose values depend on the size (extent) of the system
\end{itemize}

What is a specific property?
\begin{itemize}
  \item Extensive properties per unit mass
\end{itemize}

What is meant by "continuum"?
\begin{itemize}
  \item The continuum idealization allows us to treat properties as point functions and to assume the properties vary continually in space with no jump discontinuities.
  \item This idealization is valid as long as the size of the system we deal with is large relative to the space between the molecules.
\end{itemize}

What is equilibrium?
\begin{itemize}
  \item A state of balance
\end{itemize}

What is thermal equilibrium?
\begin{itemize}
  \item When temperature is uniform in system
\end{itemize}

What is mechanical equilibrium?
\begin{itemize}
  \item No changes in pressure anywhere in system with time
\end{itemize}

What is phase equilibrium?
\begin{itemize}
  \item When the mass of 2 phases in a system reaches equilibrium
\end{itemize}

What is chemical equilibrium?
\begin{itemize}
  \item If the chemical composition of a system does not change with time
\end{itemize}

What is a simple compressible system?
\begin{itemize}
  \item If a system involves no electrical, magnetic, gravitational, motion and surface tension effects
\end{itemize}

What is the state postulate?
\begin{itemize}
  \item The state of a simple compressible system is completely specified by 2 independent, intensive properties
\end{itemize}

What is a process?
\begin{itemize}
  \item Any change that a system undergoes from one equilibrium to another
  \item To describe a process completely, we must specify the initial and final states, as well as the path it follows, and the interactions with the surroundings
\end{itemize}

What is a path?
\begin{itemize}
  \item The series of states through which a system passes during a process
\end{itemize}

What is a Quasistatic process?
\begin{itemize}
  \item When a process proceeds in such a manner that the system remains infinitesimally close to an equilibrium state at all times
\end{itemize}

What is an isothermal process?
\begin{itemize}
  \item A process during which T remains constant
\end{itemize}

What is an isobaric process?
\begin{itemize}
  \item A process during which P remains constant
\end{itemize}

What is an isochoric process?
\begin{itemize}
  \item A process during which the specific volume v remains constant
\end{itemize}

What is a cycle?
\begin{itemize}
  \item A process during which the initial and final states are identical
\end{itemize}

What does the term "steady" mean?
\begin{itemize}
  \item No change with time
  \item The opposite of steady is transient
\end{itemize}

What is a steady-flow process?
\begin{itemize}
  \item A process during which a fluid flows through a control volume steadily
\end{itemize}

What is the 0th law of thermodynamics?
\begin{itemize}
  \item If 2 bodies are in thermal equilibrium with a 3rd body, they are also in thermal equilibrium with each other
  \item If 2 bodies have the same temperature they are in thermal equilibrium
\end{itemize}

What is the ice point?
\begin{itemize}
  \item A mixture of ice and water that is in equilibrium with air saturated with vapour at 1 atm
\end{itemize}

What is the steam point?
\begin{itemize}
  \item A mixture of liquid water and water vapour (with no air) in equilibrium at 1 atm
\end{itemize}

What is the Kelvin scale?
\begin{itemize}
  \item Temperature scale starting from absolute zero
\end{itemize}

What is pressure?
\begin{itemize}
  \item The normal force exerted by a fluid per unit area
\end{itemize}

What is absolute pressure?
\begin{itemize}
  \item Actual pressure at a given position measured against vacuum
\end{itemize}

What is Gage pressure?
\begin{itemize}
  \item Difference between absolute pressure and local atm pressure
\end{itemize}

What is vacuum pressure?
\begin{itemize}
  \item Pressures below atm
\end{itemize}
\end{document}