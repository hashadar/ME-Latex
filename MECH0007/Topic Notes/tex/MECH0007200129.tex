\documentclass[class=report, crop=false, 12pt,a4paper]{standalone}
\usepackage{enumitem}
\usepackage{gensymb}
\usepackage{siunitx}
\sisetup{detect-all}
\begin{document}
THIS IS STUFF THAT I JUST SAW ON THE BOARD WHEN I CAME IN LATE:

[...] the \(\sigma\) can quickly build up to that necessary to break the bonds at a crack tip (i.e. exceed theoretical strength). Long, deep cracks are more severe than shallower ones and sharp cracks are worse than rounded or blunt ones. Thus, on breaking A-B bond, the crack gets longer, \(\sigma\)'s concentration increases and the crack propagates uncontrollably (i.e. a brittle material).
\subsection{Fatigue strength}
The ability to resist failure under a cyclical load. Fatigue occurs at loads well below \(\sigma_{UTS}\) and can initiate and then propagate defects (principally cracks) to a point where they cause failure by fracture. Steels typically have a lifetime of \(10^6 - 10^8\) cycles for typical loads. All materials 'suffer' fatigue failure, even though the internal mechanisms vary from category to category. The amplitude of the alternating stress is a factor. Low amplitudes favour many cycles before failure - 'long lifetime' but lifetime is stress dependent.
\subsection{Other properties}
\begin{itemize}[noitemsep]
  \item Cost
  \begin{itemize}
    \item Dependent on the availability (rarity) of a material. 
    \item Manufacturing - note: polymers are expensive in but but component cost is reduced due to ease of manufacturing.
  \end{itemize}
  \item Corrosion resistance.
  \begin{itemize}
    \item Rate of reaction with an environment. Most applicable to metals but polymers and ceramics also suffer from corrosion (to a degree).
  \end{itemize}
  \item Abrasion resistance
  \begin{itemize} 
    \item The ability to avoid wear in components that are in sliding contact (bearings). 
    \item Usually, materials with high hardness is best for reducing wear but we must also consider frictional coefficients.
  \end{itemize}
  \item Environmental 'credentials'
  \begin{itemize}
    \item Recyclability and re-use. Rubber is a good example of something which is very hard to recycle.
    \item Toxicity/waste.
  \end{itemize}
\end{itemize}
\subsection{Stress/strain responses}
Many mechanical properties can either be measured, inferred or illustrated from a stress-strain response. We plot \(\sigma\) vs \(\epsilon\) in tension. Doing such a test we are measuring force and deflection and using the dimensions of the specimen we can calculate \(\sigma/\epsilon\). Extension is measured using an extensometer (laser based or contact based). The extensometer is placed across the gauge length (distance between the two necks of a dog bone shaped specimen). It measures \(L + \Delta L\). The distance between the grips is also an alternative but this is inaccurate. 

Deviation from linearity normally due to extending \(\sigma_y\) because metals are linearly elastic, caused by slip. Drop in \(\sigma\) required just before failure due to 'necking'. Local failure point, material will decrease in cross section. Necking is a random instability in the material; if at some point it gets longer faster than elsewhere. Due to the volume being conserved, specimen gets thinner. This raises the global stress applied such that further extension is focussed in this region. This is known as a positive feedback mechanism and the neck is the ultimate failure location. You can plot 'true stress' (and strain) which takes into account the decrease in cross section area. 




\end{document}