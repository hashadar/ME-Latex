\documentclass[class=report, crop=false, 12pt,a4paper]{standalone}
\usepackage{enumitem}
\begin{document}
\begin{center}
  \begin{tabular}{ |c|c|c| } 
   \hline
   Dr Wojcik & Room 407 Roberts & a.wojcik@ucl.ac.uk \\ 
   \hline
  \end{tabular}
  \end{center}

\section{Coursework and exam}
\begin{itemize}
  \item Metallography and Microstructure - Lab week (week 26), held in Materials Lab Sub Basement Roberts.
  \item Manufacturing Case Study - Friday afternoon consultancy slots (14:30 onwards), Materials Lab Sub Basement Roberts.
  \item One three hour exam, consistent of two sections: MCQ (sort of?) and conventional. Timetabled revision slots in summer term.
\end{itemize}

Add notes about the following topics: 
\begin{itemize}
  \item Cracking - welding vs riveting, arrestment of cracks in a material. 
  \item Casting and forging - in context of manufacturing processes, perocity of a liquid vs a solid.
  \item Crystalline and amorphous solids -  materials which are one or the other, both or both in different capacities, look into molecular structure of each, crystallinity is a degree of structure which affects its properties.
  \item Plasticity and deformations in metals and other materials.
\end{itemize}

Structure - Property Relationship at the heart of all Material Science.
\begin{center}
  \begin{tabular}{ |c|c|c| } 
   \hline
    \multicolumn{2}{|c|}{Manufacturing (Processing)} \\
    \hline
    Structure & Property \\
   \hline
  \end{tabular}
\end{center}

Categories of materials - these all have different structures and hence properties.
\begin{itemize}[noitemsep]
  \item Metals
  \item Ceramics
  \item Polymers
  \item Composites - a mix of the other three above (physical mix), provides an extra level of structure for scientists to work with.
\end{itemize}

\end{document}