\documentclass[11pt]{article}
\usepackage{graphicx}
\usepackage{hyperref}
%\usepackage{appendix}
\usepackage{amsmath}
\usepackage{amsthm}
\usepackage{amssymb}
\usepackage{float}
\usepackage{commath}
%\usepackage{siunitx}
%\sisetup{detect-all}
\usepackage{listings}
\usepackage{color} %red, green, blue, yellow, cyan, magenta, black, white
\definecolor{mygreen}{RGB}{28,172,0} % color values Red, Green, Blue
\definecolor{mylilas}{RGB}{170,55,241}
\usepackage[a4paper,margin=20mm]{geometry}
\numberwithin{equation}{section}
\setlength{\parskip}{\baselineskip}
\setlength{\parindent}{0pt}
\hypersetup{
    colorlinks=true,
    linkcolor=magenta,
    filecolor=magenta,      
    urlcolor=magenta,
}
\urlstyle{same}
\begin{document}
\title{\textbf{UCL Mechanical Engineering 2020/2021}\\ENGF0004 Coursework 2}
\author{NCWT3}
\maketitle
\section{Question 1}
\subsection{a}
For the line integral to be independent from the path of integration, the following conditions must be fulfilled:
\begin{gather}
    I = \int_A^B \left(\frac{\partial u}{\partial x} \dif x + \frac{\partial u}{\partial y}\dif y\right)\\
    P\left(x, \, y\right) = \frac{\partial u}{\partial x} \textrm{ and } Q\left(x, \, y\right) = \frac{\partial u}{\partial y}\\
    \frac{\partial P\left(x, \, y\right)}{\partial y} = \frac{\partial Q\left(x, \, y\right)}{\partial x}
\end{gather}
Considering the integral:
\begin{gather}
    I = \int_A^B \left[e^{-\alpha xy} \left(\frac{\alpha-2}{x}\right)\dif x - \frac{1}{\alpha y}\left(e^{-\alpha x y}-1\right)\dif y\right]\\
    P\left(x, \, y\right) = e^{-\alpha xy} \left(\frac{\alpha-2}{x}\right) \textrm{ and } Q\left(x, \, y\right) = - \frac{1}{\alpha y}\left(e^{-\alpha x y}-1\right)\\
    \frac{\partial P\left(x, \, y\right)}{\partial y} = -\alpha x \left(\frac{\alpha -2}{x}\right) e^{-\alpha xy} = \left(2\alpha - \alpha^2\right)e^{-\alpha xy}\\
    \frac{\partial Q\left(x, \, y\right)}{\partial x} = -\frac{1}{\alpha y}\left(-\alpha y\right)e^{-\alpha xy} = e^{-\alpha xy}\\
    \therefore 2\alpha e^{-\alpha xy} - \alpha^2 e^{-\alpha xy} = e^{-\alpha xy}\\
    e^{-\alpha xy} \left(\alpha^2 - 2\alpha + 1\right) = 0\\
    e^{-\alpha xy} = 0 \rightarrow \textrm{no solutions}\\
    \left(\alpha - 1\right)^2 = 0\\
    \alpha = 1
\end{gather}
\subsection{b}
Calculating the line integral of \ref{1beq1} from $O\left(0, \, 0\right)$ to $A(1, \, e-1)$ along $y=e^x -1$:
\begin{gather}
    I = \int_O^A \left(ye^{-2x}\right)\left(\dif x + \dif y\right) \label{1beq1}\\
    y = e^x - 1\\
    \dif y = e^x \dif x\\
    I = \int_0^1 \left( \left( e^x-1\right) \left( e^{-2x} \right) + \left( e^x-1\right)\left( e^{-2x}\right)\left( e^x\right)\right)\dif x \\
    = \int_0^1 \left(e^{-x} - e^{-x} - e^{-2x} +1\right)\dif x\\
    = \int_0^1 \left(1-e^{-2x}\right)\dif x \\
    = \left[x + \frac{e^{-2x}}{2}\right]_0^1\\
    = 1 + \frac{e^{-2}}{2} -0-\frac{1}{2}\\
    I = \frac{1}{2}\left(e^{-2}+1\right)
\end{gather}
\subsection{c}
\subsubsection{i}
\begin{gather}
    \underline{F}\left(x, \, y, \, z\right) = \begin{pmatrix}
        \frac{y}{x^2}\\
        \frac{x}{y^2}
    \end{pmatrix}\\
    \nabla \cdot \underline{F} = \begin{pmatrix}
        \frac{\partial}{\partial x}\\
        \frac{\partial}{\partial y}
    \end{pmatrix} \cdot \begin{pmatrix}
        \frac{y}{x^2}\\
        \frac{x}{y^2}
    \end{pmatrix}\\
    = \frac{\partial}{\partial x}\left(\frac{y}{x^2}\right) + \frac{\partial}{\partial y} \left(\frac{x}{y^2}\right)\\
    = -\frac{2y}{x^3}-\frac{2x}{y^3} \\
    = -2\left(\frac{y}{x^3}+\frac{x}{y^3}\right)
\end{gather}
\subsubsection{ii}
\begin{gather}
    I = \int_{1}^{2} \int_{1}^{2} \left(-2\left(\frac{y}{x^3}+\frac{x}{y^3}\right)\right) \dif x \dif y\\
    = \int_{1}^{2} \left[-2\left(\frac{y}{-2x^2} + \frac{x^2}{2y^3}\right)\right]_1^2 \dif y\\
    = \int_{1}^{2} \left[-2\left(-\frac{y}{8} + \frac{2}{y^3}+\frac{y}{2} - \frac{1}{2y^3}\right)\right] \dif y\\
    = \int_{1}^{2} \left(-\frac{3y}{4}-\frac{3}{y^3}\right) \dif y\\
    = \left[-\frac{3y^2}{8}+\frac{3}{2y^2}\right]_1^2\\
    = -\frac{3}{2} + \frac{3}{8} + \frac{3}{8} - \frac{3}{2}\\
    I = -\frac{9}{4}
\end{gather}
\subsection{d}
\subsubsection{i}
\begin{gather}
    I = \int \left(\sin x \cos y \dif y + \cos x \sin y \dif y\right)\\
    y = 0 \hspace{1cm} \dif y = 0\\
    I_{AB} = \int_{x=0}^{\pi} \left(\sin x\right)\dif x = \left[-\cos x\right]_0^{\pi} = 2\\
    x = \pi \hspace{1cm} \dif x = 0\\
    I_{BC} = \int_{y=0}^{\pi} \left(-\sin y\right)\dif y= \left[\cos y\right]_0^{\pi} = -2\\
    \therefore I = I_{AB} + I_{BC} = 2-2 = 0
\end{gather}
\subsubsection{ii}
\begin{gather}
    I = \int \left(\sin x \cos y \dif y + \cos x \sin y \dif y\right)\\
    y = x \hspace{1cm} \dif y = \dif x\\
    I_{AC} = \int_0^{\pi} \left(\sin x \cos x + \sin x \cos x\right) \dif x\\
    = \int_0^{\pi} \left(\sin \left(2x\right)\right)\dif x\\
    I_{AC} = \left[-\frac{1}{2}\cos \left(2x\right)\right]_0^{\pi} = \frac{1}{2} - \frac{1}{2} = 0\\
\end{gather}
\end{document}