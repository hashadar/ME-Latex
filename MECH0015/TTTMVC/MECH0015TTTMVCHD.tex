\documentclass[11pt]{article}
\usepackage{graphicx}
\usepackage{cite}
\def\BibTeX{{\rm B\kern-.05em{\sc i\kern-.025em b}\kern-.08em
    T\kern-.1667em\lower.7ex\hbox{E}\kern-.125emX}}
\usepackage{url}
    \makeatletter
    \g@addto@macro{\UrlBreaks}{\UrlOrds}
    \makeatother
\usepackage{appendix}
\usepackage{amsmath}
\usepackage{booktabs}
\renewcommand{\arraystretch}{1.2}
\usepackage{amssymb}
\usepackage{float}
\usepackage{commath}
\usepackage{siunitx}
\usepackage{multirow}
\usepackage[a4paper,margin=20mm]{geometry}
\setlength{\parskip}{\baselineskip}%
\setlength{\parindent}{0pt}%
\sisetup{detect-all}
\begin{document}
\title{\textbf{UCL Mechanical Engineering}\\MECH0015 TTT \& MVC Quiz Questions}
\author{Hasha Dar}
\maketitle
\section{Why does the tensile strength of a steel drop with carbon content after the eutectoid composition is reached?}
\section{Shw that the area under the stress strain curve is equivalent to the elastic stored energy (in a stressed material). What part of the stress strain curve do you need?}
\section{Why is a process anneal (or stress relief anneal) not usually applied to steels above about 0.4\% carbon content? }
\section{Which way do we connect up the power supply in an impressed current cathodic protection scheme and why - and why should we still consider painting the buried pipelines as well (in addition to cathodically protecting it)?}
\end{document}