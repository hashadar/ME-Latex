%%%%%%%%%%%%%%%%%%%%%%%%%%%%%%%%%%%%%%%%%
% Medium Length Graduate Curriculum Vitae
% LaTeX Template
% Version 1.1 (9/12/12)
%
% This template has been downloaded from:
% http://www.LaTeXTemplates.com
%
% Original author:
% Rensselaer Polytechnic Institute (http://www.rpi.edu/dept/arc/training/latex/resumes/)
%
% Important note:
% This template requires the res.cls file to be in the same directory as the
% .tex file. The res.cls file provides the resume style used for structuring the
% document.
%
%%%%%%%%%%%%%%%%%%%%%%%%%%%%%%%%%%%%%%%%%

%----------------------------------------------------------------------------------------
%	PACKAGES AND OTHER DOCUMENT CONFIGURATIONS
%----------------------------------------------------------------------------------------

\documentclass[margin, 10pt]{res} % Use the res.cls style, the font size can be changed to 11pt or 12pt here

\usepackage{helvet} % Default font is the helvetica postscript font
\renewcommand{\familydefault}{\sfdefault}
%\usepackage{newcent} % To change the default font to the new century schoolbook postscript font uncomment this line and comment the one above

\usepackage{siunitx}

\setlength{\textwidth}{5.1in} % Text width of the document

\usepackage{enumitem}

\newlength\mylen
\settowidth\mylen{\textbullet}
\addtolength\mylen{-3mm}
\setlist[itemize,1]{leftmargin=*,labelsep=-\mylen}

\begin{document}

%----------------------------------------------------------------------------------------
%	NAME AND ADDRESS SECTION
%----------------------------------------------------------------------------------------

\moveleft.5\hoffset\centerline{\large\bf Hasha Humayon Dar} % Your name at the top
 
\moveleft\hoffset\vbox{\hrule width\resumewidth height 1pt}\smallskip % Horizontal line after name; adjust line thickness by changing the '1pt'

\moveleft.5\hoffset\centerline{Postadresse: 17 Manor road, Chigwell, Essex, IG7 5PF}
\moveleft.5\hoffset\centerline{Handynummer: +44 73055 76221}
\moveleft.5\hoffset\centerline{E-Mail: darhasha5@gmail.com} % Your address
\moveleft.5\hoffset\centerline{Webseite: hashadar.com}

%----------------------------------------------------------------------------------------

\begin{resume}

%----------------------------------------------------------------------------------------
%	EDUCATION SECTION
%----------------------------------------------------------------------------------------
 
\section{Ausbildung}  

\textbf{Maschinenbau MEng, University College London \hfill 2019 bis heute}

Schwerpunktmodule: Mathematical Modelling \& Analysis, Fundamentals of Materials, 
Engineering Dynamics, Introduction to Thermodynamics and Fluid Dynamics,
Mechanical Engineering Practical Skills.

\textbf{A-Levels, Bancroft's School \hfill 2017 - 2019}

Mathematics A*, Further Mathematics - A, Physics - A, German - B

\textbf{GCSEs, Bancroft's School
\hfill 2015 - 2017}

4 A*, 4 A, 2 B

%----------------------------------------------------------------------------------------
%	EXPERIENCE SECTION
%----------------------------------------------------------------------------------------

\section{RELEVANTE \\ ERFAHRUNG} 

\textbf{University College London Formel Student Rennfahrer-Team \hfill 2019 bis heute}
\\
\begin{itemize}
  \item \textbf{Forschung:} Unterstützung des Aerodynamikführers bei der Auswahl und Konfiguration des Aerodynamikpakets des Fahrzeugs.
  \item \textbf{Stückliste:} arbeitete an der Kalkulation und Stückliste für den Rennwagen.
  \item \textbf{Elektrisches Design:} Unterstützung beim Entwurf elektrischer Schaltkreise für Komponenten des Rennwagens.
\end{itemize}

\textbf{UCL Drohnen-Team \hfill 2020 - present}
\\
\begin{itemize}
  \item \textbf{Führung:} Gruppenleiter für das Projekt mit 4 anderen
  \item \textbf{Fusion 360} wurde verwendet, um die Drohne zu entwerfen
\end{itemize}


%----------------------------------------------------------------------------------------
%	ENGINEERING PROJECTS SECTION
%----------------------------------------------------------------------------------------

\section{INGENIEUR- \\ PROJEKTE}

\textbf{IMechE Challenge - Squashball Linienwerfer: \\Entwerfen, bauen und testen \hfill 2020}
\\
\begin{itemize}
  \item \textbf{Kommunikation:} 
  Als Teamleiter arbeitete ich mit Aufgaben zusammen und verteilte sie effektiv auf mein Team. Ich spielte mit den verschiedenen Stärken meines Teams und stellt Fristen ein.  
  \item \textbf{Design:} entwarf einen Linienwerfer in einem Sechserteam, um einen Squashball (40\si{\milli \meter}) mit einer Linie abzufeuern, die an einem Ziel mit einem Durchmesser von 120-490 \si{\milli \meter} befestigt ist. Budgetiert, um nicht mehr als 25 Pfund zu kosten und kaufte gekaufte Teile sowie proprietäre Teile.
  \item \textbf{IT:} hat \textbf{CATIA V5} verwendet, um unseren ersten Prototyp-Linienstarter zu erstellen. \textbf{MATLAB} wurde auch für Berechnungen verwendet.
  \item \textbf{Präsentation:} Ein 30-seitiges Design-Portfolio und eine PowerPoint-Präsentation wurden erstellt, um einem Panel von fünf technischen Experten präsentiert zu werden. Fortsetzung nächste Seite.
\end{itemize}
 
%----------------------------------------------------------------------------------------
%	OTHER EXPERIENCE SECTION
%----------------------------------------------------------------------------------------
 
\section{ANDERE \\ ERFAHRUNG}
\textbf{Freischaffender Fotograf \hfill 2016 bis heute}
\\
\begin{itemize}
  \item \textbf{Führung:} Die freiberufliche Tätigkeit hat mir verschiedene Fähigkeiten beigebracht, wie z. B. Zeitmanagement und effektive Geschäftsstrategien.
  \item \textbf{technische Fähigkeiten:} Ich habe Erfahrung mit Autodidakten Kameratechniken und verschiedene Fotografie-Stile einschließlich Studio-, Mode- und Event-Fotografie.
  \item \textbf{Kreatives Portfolio:} Dies kann unter hashadar.com eingesehen werden
\end{itemize}

\textbf{Theater Sound Designer \hfill 2017 bis heute}
\\
\begin{itemize}
  \item \textbf{Management:} Ich habe als leitender Sounddesigner für viele Theateraufführungen an Universitäten und Schulen gearbeitet und mich in große Teams mit Bühnencrew, Besetzung und Regisseuren integriert.
  \item \textbf{Technische Fähigkeiten:} Ich habe an einer Schulung bei University College London Stage Crew teilgenommen, um effektive und gut strukturierte integrierte Sounddesigns für zahlreiche Shows zu erstellen.
\end{itemize}

\textbf{Privatlehrer \hfill 2019 bis heute}
\\
\begin{itemize}
  \item \textbf{Management:} Durch die Verwaltung mehrerer Studenten rund um meinen Stundenplan habe ich Disziplin und Zeitmanagement gelernt. Ein Gleichgewicht zu finden ist von größter Bedeutung, um nicht ins Hintertreffen zu geraten.
  \item \textbf{Kommunikation:} Jüngeren Schülern komplexe mathematische und physikalische Konzepte zu erklären, erfordert Geduld und Witz. Es ist wichtig, Wege zu finden, um bestimmte Konzepte anders zu erklären, wenn die Schüler Schwierigkeiten haben, sie zu verstehen.
\end{itemize}

%----------------------------------------------------------------------------------------
% SOFTWARE SECTION
%----------------------------------------------------------------------------------------

\section{SOFTWARE}
\textbf{MATLAB}\\
\textbf{CAD:} CATIA V5 und Autodesk Fusion 360.\\
\textbf{Microsoft Office:} Word, PowerPoint, Excel, Teams.\\
\textbf{LaTeX:} Meine Arbeit kann unter github.com/hashadar angesehen werden

%----------------------------------------------------------------------------------------
% INTERESTS SECTION
%----------------------------------------------------------------------------------------

\section{INTERESSEN} 
\textbf{Sprachen:} Muttersprache in Englisch und Urdu. Konversation auf Deutsch (B2).\\
\textbf{Sport:} Fechter und Badmintonspieler. Spielte 2016 und 2017 an nationalen Fechtwettbewerben. Kapitän des Redbridge Fencing Teams und des Bancroft Fencing Teams 2015-2018.\\
\textbf{Leser:} Regelmäßiger Leser von Technologie-Nachrichten und aktuellen Angelegenheiten.

%----------------------------------------------------------------------------------------
Hinweis: Da Deutsch nicht meine Muttersprache ist, entschuldige ich mich aufrichtig für Fehler oder Fehlübersetzungen.

\end{resume}
\end{document}